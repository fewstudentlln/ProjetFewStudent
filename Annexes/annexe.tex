\documentclass[12pt,a4paper]{article}

% Encodage et langue
\usepackage[utf8]{inputenc}
\usepackage[T1]{fontenc}
\usepackage[french]{babel}

% Liens cliquables
\usepackage[colorlinks=true, linkcolor=blue, urlcolor=blue]{hyperref}

% Mise en page simple
\usepackage{geometry}
\geometry{
    left=2.5cm,
    right=2.5cm,
    top=2.5cm,
    bottom=2.5cm
}

\title{Annexe}
\author{Projet Formula Student}
\date{\today}

\begin{document}

\maketitle

\section*{Liens utiles}

Voici une liste de liens vers différents sites utiles pour le projet.  
Cliquez sur le texte pour ouvrir le site correspondant.

\begin{itemize}
    \item \textbf{Site officiel UCLouvain} : \href{https://www.uclouvain.be}{https://www.uclouvain.be}
    \item \textbf{Page Formula Student} : \href{https://osf.imeche.org/sitefinity/status?ReturnUrl=https:%2F%2Fosf.imeche.org%2Fevents%2Fformula-student}{https://www.formulastudent.de}
    \item \textbf{Overleaf} : \href{https://www.overleaf.com}{https://www.overleaf.com}
    \item \textbf{GitHub EPL Team} : \href{https://github.com/fewstudentlln}{https://github.com/fewstudentlln}
\end{itemize}

% Exemple de section supplémentaire si vous voulez ajouter plus de liens
\section*{Autres ressources}

\begin{itemize}
    \item \textbf{Documentation LaTeX} : \href{https://www.latex-project.org}{https://www.latex-project.org}
    \item \textbf{Wikipedia Formula Student} : \href{https://en.wikipedia.org/wiki/Formula_Student}{\url{https://en.wikipedia.org/wiki/Formula_Student}}
\end{itemize}

\end{document}
