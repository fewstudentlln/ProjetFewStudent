\documentclass[12pt, a4paper]{report}

% --- Packaged de base ---
\usepackage[utf8]{inputenc}    % Encodage
\usepackage[T1]{fontenc}       % Encodage des polices
\usepackage[french]{babel}     % Langue française
\usepackage{graphicx}          % Insertion d'images
\usepackage{geometry}          % Gestion des marges
\geometry{top=2.5cm, bottom=2.5cm, left=2.5cm, right=2.5cm}

% --- Esthétique et Navigation ---
\usepackage{fancyhdr}          % En-têtes et pieds de page
\usepackage[hidelinks]{hyperref}          % Liens cliquables
\usepackage{booktabs}          % Tableaux professionnels

% --- Configuration de l'en-tête ---
\pagestyle{fancy}
\fancyhf{}
\lhead{Projet Formula Student}
\rhead{Nom de l'Écurie}
\cfoot{\thepage}

% --- Début du document ---
\begin{document}

% --- Page de Garde ---
\begin{titlepage}
    \centering
    \vspace*{1cm}
    {\huge\bfseries Dossier de Création d'Équipe \par}
    \vspace{0.5cm}
    {\LARGE Formula Student \par}
    \vspace{1.5cm}
    \includegraphics[width=0.6\textwidth]{PNG/logo_projet.png}\par % Remplacez par votre logo
    \vspace{1.5cm}
    {\large Préparé par : \textbf{Prénom Nom} \par}
    \vfill
    {\large Date : \today \par}
\end{titlepage}

% --- Sommaire ---
\tableofcontents
\newpage

% --- Chapitre 1 : Vision ---
\chapter{Présentation du Projetssss}
\section{Vision et Objectifs}
L'objectif est de concevoir une monoplace de type Formula Student pour participer aux compétitions européennes.

\section{Valeurs de l'Équipe}
Innovation, rigueur technique et esprit d'équipe.

% --- Chapitre 2 : Organisation ---
\chapter{Management et Organisation}
\section{Structure de l'Équipe}
L'équipe est divisée en plusieurs pôles techniques et administratifs :
\begin{itemize}
    \item Pôle Châssis et Liaison au sol.
    \item Pôle Powertrain (Électrique/Thermique).
    \item Pôle Sponsoring et Communication.
\end{itemize}

% --- Chapitre 3 : Étude Technique ---
\chapter{Conception Technique}
\section{Spécifications du Véhicule}
Description des cibles de performance (poids, puissance, aérodynamisme).

% --- Chapitre 4 : Budget et Financement ---
\chapter{Business Plan}
\section{Budget Prévisionnel}
\begin{table}[h]
    \centering
    \begin{tabular}{lr}
        \toprule
        Poste de dépense & Montant estimé (€) \\
        \midrule
        Matières premières & 5 000 \\
        Électronique & 3 000 \\
        Inscription compétitions & 2 500 \\
        \bottomrule
    \end{tabular}
    \caption{Estimation budgétaire préliminaire}
\end{table}

% --- Conclusion ---
\chapter*{Conclusion}
\addcontentsline{toc}{chapter}{Conclusion}
Ce projet représente un défi d'ingénierie majeur pour notre établissement...

\end{document}